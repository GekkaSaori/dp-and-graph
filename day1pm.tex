\documentclass{article}
\usepackage{ctex}
\usepackage{amsmath}
\usepackage{minted}
\DeclareMathOperator{\band}{and}
\begin{document}
\section{Classical Problems}
\subsection{最长上升子序列}
二分优化掉内层循环。注意到LIS的一个性质,如果当前的数比之前的某个数大,那么它的LIS长度比之前的那个数大。注意到之前$O(n^2)$的算法的内层循环的问题。

明显地我们可以使用二分技术优化内层循环,维护一个数组g[],使得它单调,然后二分查找它的LIS的长度。
\subsection{滑雪}

dp[i][j]表示到(i,j)这个格子的最长滑雪路线长度。

用推的方法:$$dp[x][y]\to dp[i][j]|x-x'|+|y-y'|=1,a[x'][y']<a[x][y]$$
按照所有点的高度进行排序,然后按照高度由高到低枚举即可。
\subsection{最长不互斥子序列}
给定一个序列,找出最长不互斥子序列,即$b[i]\band b[i-1]\neq 0.$

类似LIS的做法,维护一个g[]数组,使它表示满足不互斥性质的子序列的长度。
\subsection{回文串划分}
给一个字符串,划分成最少个回文子串。长度不超过1000.

令$dp[i]$表示已经将字符串的第1到i位处理完毕的最少划分次数。

$$dp[i]=\min\{dp[j]+1\},j<i,s[j,i]是回文串.$$
使用字符串Hash算法,判断是否是回文串(只需要对一个字符串正过来做一半的Hash,倒着再做一遍Hash,判断Hash是否相等。)
\begin{itemize}
    \item{若回文串长度为奇数,可以预处理一个数组,表示以i点为中心的最长回文串长度 => $O(n^2)$.}
    \item{若回文串长度为偶数,额外处理一个数组,表示以i和i+1为中心的最长回文串长度(使得a[i]=a[i+1],a[i-1]=a[i-2])=>$O(n^2)$.}
\end{itemize}
\subsection{传球问题}
有N个人排成一个环,每个人选择向左或向右传球,最后一个拿到球的人输。问游戏进行M轮,第i个人输的方案数是多少。N不超过30,M不超过30。

dp[i,x]表示经过了i轮,第x个人拿到球的方案数。
\begin{equation*}
    \begin{aligned}
        dp[i,x]=\min\{dp[i-1,p]+C(p,r)\}\\
        dp[i][x]=\sum_{(p,r)} dp[i-1,p]
    \end{aligned}
\end{equation*}
\subsubsection{阶梯序列}
B序列是梯子序列,当且仅当:
存在$x$使得
\begin{equation*}
    B(1)\le B(2)\le\cdots\le B(x)\ge B(x+1)\ge\cdots\ge B(N).
\end{equation*}

给定一个序列A,有Q次询问$A(L\ldots R)$是不是梯子序列.

$N,Q\le 10^5$.

做一遍最长不降子序列+最长上升子序列。

预处理up[i]表示以i为起点向左最大有多少个单调上升的数,同样预处理dowm[i]数组表示向右有多少个单调上升的数,然后判断一下区间[L,R]内的up与down的长度。

\subsection{区间染色}
给定一个长度为N的序列,每一位有一个目标颜色,每次可以选择一个区间,将区间内的所有元素改为其目标颜色。设区间内不同颜色的数量为X,则操作的代价为$X^2$。求最小代价。$N\le 5\times 10^4.$ 注意该点需要什么颜色就必须染成什么颜色,不能染成别的颜色。

\end{document}
