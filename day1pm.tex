\documentclass{article}
\usepackage{ctex}
\usepackage{amsmath}
\usepackage{minted}
\DeclareMathOperator{\band}{and}
\begin{document}
\section{Classical Problems}
\subsection{最长上升子序列}
二分优化掉内层循环。注意到LIS的一个性质,如果当前的数比之前的某个数大,那么它的LIS长度比之前的那个数大。注意到之前$O(n^2)$的算法的内层循环的问题。

明显地我们可以使用二分技术优化内层循环,维护一个数组g[],使得它单调,然后二分查找它的LIS的长度。
\subsection{滑雪}

dp[i][j]表示到(i,j)这个格子的最长滑雪路线长度。

用推的方法:$$dp[x][y]\to dp[i][j]|x-x'|+|y-y'|=1,a[x'][y']<a[x][y]$$
按照所有点的高度进行排序,然后按照高度由高到低枚举即可。
\begin{minted}{C++}
#include <cstdlib>
#include <cstdio>
#include <cstring>
#include <cmath>

#include <algorithm>
#include <vector>
#include <utility>

using namespace std;

const int dx[] = {-1, 1, 0, 0};
const int dy[] = {0, 0, -1, 1};

void update(int &a, int b) {
    if (a < b) {
        a = b;
    }
}

int a[100007];
int dp[100007];
int n, m, ans;

vector<pair<int, int> > b;

bool cmp(const pair<int, int> &x, const pair<int, int> &y) {
    return a[x.first * m + x.second] > a[y.first * m + y.second];
}

int main(void) {
    scanf("%d%d", &n, &m);
    for (int i = 0; i < n; ++i) {
        for (int j = 0; j < m; ++j) {
            scanf("%d", a + (i*m+j));
            b.push_back(make_pair(i, j));
        }
    }

    sort(b.begin(), b.end(), cmp);
    for (int i = 0; i < b.size(); ++i) {
        int x = b[i].first, y = b[i].second;
        update(ans, dp[x * m + y]);
        for (int k = 0; k < 4; ++k) {
            int x1 = x + dx[k], y1 = y + dy[k];
            if (x1 >= 0 && x1 < n && y1 >= 0 && y1 < m) {
                if (a[x1 * m + y1] < a[x * m + y]) {
                    update(dp[x1 * m + y1], dp[x * m + y] + 1);
                }
            }
        }
    }

    return 0;
}
\end{minted}
\subsection{最长不互斥子序列}
给定一个序列,找出最长不互斥子序列,即$b[i]\band b[i-1]\neq 0.$

类似LIS的做法,维护一个g[]数组,使它表示满足不互斥性质的子序列的长度。
\begin{minted}{C++}
#include <cstdlib>
#include <cstdio>
#include <cstring>
#include <cmath>

int n, a[100007], dp[100007], g[32];
int ans;

void update(int &a, int b) {
    if (a < b) a = b;
}

int main(void) {
    scanf("%d", &n);
    for (int i = 0; i < n; ++i) {
        scanf("%d", a+i);
    }
    for (int i = 0; i < n; ++i) {
        int temp = 0;
        for (int k = 0; k < 31; ++k) {
            if (a[i] & (1 << k)) {
                update(temp, g[k]);
            }
        }
        dp[i] = temp + 1;
        for (int k = 0; k < 31; ++k) {
            if (a[i] & (1 << k)) {
                update(g[k], dp[i]);
            }
        }

        update(ans, dp[i]);
    }
    printf("%d\n", ans);

    return 0;
}
\end{minted}
\subsection{回文串划分}
给一个字符串,划分成最少个回文子串。长度不超过1000.

令$dp[i]$表示已经将字符串的第1到i位处理完毕的最少划分次数。

$$dp[i]=\min\{dp[j]+1\},j<i,s[j,i]是回文串.$$
使用字符串Hash算法,判断是否是回文串(只需要对一个字符串正过来做一半的Hash,倒着再做一遍Hash,判断Hash是否相等。)

or:

\begin{itemize}
    \item{若回文串长度为奇数,可以预处理一个数组,表示以i点为中心的最长回文串长度 => $O(n^2)$.}
    \item{若回文串长度为偶数:}
    \begin{itemize}
        \item{额外处理一个数组,表示以i和i+1为中心的最长回文串长度(使得a[i]=a[i+1],a[i-1]=a[i-2])=>$O(n^2)$.}
        \item{在两个字符之间都插入一个特殊字符,然后这个串的长度就变成了奇数,再按照长度为奇数的方法做.}
    \end{itemize}
\end{itemize}
\begin{minted}{C++}
#include <cstdlib>
#include <cstdio>
#include <cstring>
#include <cmath>

char s[100007];
int a[100007], b[100007];
int n;
int dp_[100007], *dp = dp_ + 1;

void update(int &a, int b) {
    if (a > b) {
        a = b;
    }
}

void init(void) {
    for (int i = 0; i < n; ++i) {
        int x = i - 1, y = i + 1;
        while (x >= 0 && y < n && s[x] == s[y]) {
            --x; ++y;
        }
        a[i] = (i - x);
    }
    for (int i = 0; i < n; ++i) {
        int x = i, y = i + 1;
        while (x >= 0 && y < n && s[x] == s[y]) {
            --x; ++y;
        }
        b[i] = (i - x);
    }
}

bool is_para(int x, int y) {
    int t = x + y;
    if (t % 2) {
        int ext = b[(t / 2)];
        return ext >= (y - x + 1) / 2;
    } else {
        int ext = a[(t / 2)];
        return ext >= (y - x) / 2 + 1;
    }
}

void work(void) {
    for (int i = 0; i < n; ++i) {
        dp[i] = n + 1;
    }
    dp[-1] = 0;
    for (int i = 0; i < n; ++i) {
        for (int j = -1; j < i; ++j) {
            if (is_para(j+1, i)) {
                update(dp[i], dp[j] + 1);
            }
        }
    }
}

int main(void) {
    scanf("%s", s);
    n = strlen(s);
    init();
    work();
    printf("%d\n", dp[n-1]);
    return 0;
}
\end{minted}
\subsection{传球问题}
有N个人排成一个环,每个人选择向左或向右传球,最后一个拿到球的人输。问游戏进行M轮,第i个人输的方案数是多少。N不超过30,M不超过30。

dp[i,x]表示经过了i轮,第x个人拿到球的方案数。
\begin{equation*}
    \begin{aligned}
        &dp[i,x]&=&\min\{dp[i-1,p]+C(p,r)\}\\
        &dp[i][x]&=&\sum_{(p,r)} dp[i-1,p]
    \end{aligned}
\end{equation*}
\subsection{阶梯序列}
B序列是梯子序列,当且仅当:
存在$x$使得
\begin{equation*}
    B(1)\le B(2)\le\cdots\le B(x)\ge B(x+1)\ge\cdots\ge B(N).
\end{equation*}

给定一个序列A,有Q次询问$A(L\ldots R)$是不是梯子序列.

$N,Q\le 10^5$.

做一遍最长不降子序列+最长上升子序列。

预处理up[i]表示以i为起点向左最大有多少个单调上升的数,同样预处理dowm[i]数组表示向右有多少个单调上升的数,然后判断一下区间[L,R]内的up与down的长度。

\subsection{区间染色}
给定一个长度为N的序列,每一位有一个目标颜色,每次可以选择一个区间,将区间内的所有元素改为其目标颜色。设区间内不同颜色的数量为X,则操作的代价为$X^2$。求最小代价。$N\le 5\times 10^4.$ 注意该点需要什么颜色就必须染成什么颜色,不能染成别的颜色。

很容易写出状态转移方程:
\begin{equation*}
    f[i]=\min\{f[j]+cost(j,i)\},j<i;
\end{equation*}
1D1D动态规划标准DP模型。

注意到直接输出n可以暴力骗分。

可以优化这个状态转移方程为$f[i]=\min(f[g[i][x]]+x^2)$,其中$g[i][x]$是记录前i个方格中有x种颜色。

g数组的求法:对于数组元素$g[i][x]$,我们有以下状态转移方程:
\begin{equation*}
    \begin{aligned}
        &g[i+1][x]&=&g[i][x]&,第i+1个方格的颜色是前面所包含的;\\
        &g[i][x]&=&g[i+1][x]&,第i+1个方格的颜色不是前面所包含的.
    \end{aligned}
\end{equation*}
\end{document}
