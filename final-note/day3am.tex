\documentclass{article}
\usepackage{ctex}
\usepackage{amsmath}
\usepackage{amssymb}
\usepackage{minted}
\usepackage[colorlinks, linkcolor=blue]{hyperref}
\usepackage{graphicx}
\newcommand{\romannum}[1]{\uppercase\expandafter{\romannumeral#1}}
\DeclareMathOperator{\bor}{or}
\setminted[C++]{autogobble,breaklines,fontsize=\footnotesize,mathescape}
\begin{document}
\title{Note on Dynamic Programming (\romannum{3})\\\large{Day 3 AM to Day 3 NIGHT---Advanced Models}}\date{}\author{Hatsune Miku}
\maketitle
\tableofcontents
\newpage

\section{“第一”计数模型}
假设进行一场OI考试,两个人拿了300分,他们都是第一名,则下一个人就是第三名。同时$a=b>c$与$b=a>c$是同一种方案。
\subsection{赛马名次}
两个人比赛的结果有三种:输、赢、平。求N个人比赛的结果。$N\le 1000.$

\begin{equation*}
    dp[i]=\sum_{j=1}^i C_i^j\times dp[i-j],dp[0]=1.
\end{equation*}

先考虑有多少个第一名,假设有$i$个人参加比赛,那么i个人参加比赛的不同方案数=i个人中有1个人作为第一名的方案数$\times$其他人接着排的方案数。

预处理出一个杨辉三角(表示出$C_n^k$),递推关系式:
\begin{equation*}
    C_n^k=C_{n-1}^k+C_{n-1}^{k-1}.
\end{equation*}
时间复杂度$O(n^2)$.
\subsection{排列问题}
有两种骨牌A和B,有无限个。需要选出n个排成一行,但要求至少有三个A被摆放在一起。$N\le 100.$

考虑枚举连续的三个A第一次出现的位置。设$f[i]$为长度为i的序列有多少个满足要求的摆放方案,则枚举三个A第一次出现的位置,设前面部分长度为$j-1$,则后面部分长度为$n-(j+2)$,前面后面都可以任意放置。后面一段摆放方案数有$2^{n-(j+2)}$种。对于前半部分,我们有$2^{j-1}-f[j-1]$种方案,总的来看,我们有$f[i]=\sum^j(2^{j-1}-f[j-1])(2^{n-(j+2)})$。有一个问题,显然我们有当前面最后一位出现的是A,然后就不符合我们枚举第一个出现的位置。所以我们强制第一位为B,则递推式修正为$f[i]=\sum^j(2^{j-2}-f[j-2])(2^{n-(j+2)})+2^{n-3}$.
\section{树模型}
树模型的基本定义:无向无环图、有$n-1$条边的图。

基本性质:树上任意两点都是相互联通的,且只有一条唯一的路径。
\subsection{直径}
无根树与有根树。显然树上最远的两个节点一点是两个叶子节点。对于一个无根树,可以指定一个节点作为根节点将这棵无根树树变为有根树,即把这棵树“拎”起来的过程。求一张图上的最短路可以在现实物理世界中做,将所有的点按照图上的边权连上一条等长的线,然后将这张图拎起来,则所有垂着的边就是最短路上的边。

可以使用标准的图论算法,使用一次DFS,记录DFS序。

已知,树上的路径只有两种情况:
\begin{enumerate}
    \item{从某个点出发形成一条路径一直向下走}
    \begin{itemize}
        \item{定义f[i]表示从i这个点往下走最远到达的点是什么}
        \item{枚举一个j点,j是i的儿子,则有$f[i]=\max\{f[j]\}+1$}.
    \end{itemize}
    \item{这条路径会在某一个点转弯}
    \begin{itemize}
        \item{类似上述的转移,算出每个点出发到达的最长链和次长链}
    \end{itemize}
\end{enumerate}
代码:
\begin{minted}{C++}
# include <cstdio>
# include <iostream>
# include <cstring>
# include <cmath>

struct sol_t {
    int a, b;
    void update(int c) {
        if(c > a) b = a, a = c;
        else if(c > b) b = c;
    }
};

struct edge_t {
    int a, b;
    edge_t *next;
};

int n;
sol_t dp[100005];
edge_t edges[100005 * 2], *first_edge[100005], *pedges = edges;

void add_edge(int a, int b) {
    edge_t *p = ++pedges;
    p->a = a, p->b = b, p->next = first_edge[a], first_edge[a] = p;
}

void work(int x, int fa) {
    for(edge_t *e = first_edge[x]; e; e = e->next) if(e->b != fa) {
        int y = e->b;

        work(y, x);
        dp[x].update((dp[y].a + 1));
    }
}

void update(int &a, int b) {
    if(a < b) a = b;
}

int main(void) {
    scanf("%d", &n);
    for(int i = 0, a, b; i < n; i++) {
        scanf("%d%d", &a, &b);
        add_edge(a, b);
        add_edge(b, a);
    }
    work(1, 0);
    int ans = 0;
    for (int i = 1; i <= n; i++)  update(ans, dp[i].a);
    for (int i = 1; i <= n; i++)  update(ans, dp[i].a + dp[i].b);
    printf("%d\n", ans);
    return 0;
}
\end{minted}

\subsection{重心}
N个节点的树,询问树的重心。重心是指删除一个点后使联通块大小的最大值最小,$N\le 100000$。

直接枚举这棵树的重心是什么。假设切掉了一个点,那么联通块分为两类,第一类为以它父亲节点为一个叶子结点的子树,另外一类为以这个节点为根节点的子树的大小。两种类型是互补的,可以知一求一。最后两者之间取一个min即可。
\subsection{奖励}
N个的关系图是棵树,树上的边代表了“上司——下属”关系。

每个人要么受到来自上司奖励的1000元,要么给予自己一个下属奖励1000元。

问所有人拿到奖励和的最大值,并输出方案。$N\le 5\times 10^5$.

对于叶子结点,只能收钱,需要排除的是在中间节点的一个人不能既发钱又收钱。

同时DP两个数组f[i]和g[i],令f[i]表示i的子树已经处理完毕,且i这个人没有收到上司的奖励。g[i]表示i的子树已经处理完毕,且i这个人收到了上司的奖励。
\begin{equation*}
    状态转移方程:
    \begin{cases}
        \begin{aligned}
            f[i]&=\max
            \begin{cases}
                \sum_jf[j],\\
                \max g[j]+\sum_{k\in j}f[k],
            \end{cases}
            \\
            g[i]&=\sum_jf[j].
        \end{aligned}
    \end{cases}
\end{equation*}
最后输出答案时,只需要再做一次DFS,记录一下每个点是从哪个点来的,递归处理不同的数组即可。
