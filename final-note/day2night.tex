区间染色代码实现:
\begin{minted}{C++}
# include <cstdlib>
# include <cstdio>
# include <cmath>
# include <cstring>

int n=0;

struct person_t
{
	float p_;
	person_t(float pp=0):p_(pp)
	{

	}
	float p (void)
	{
		return p_;
	};
	float q(void)
	{
		return 1-p_;
	};
};


person_t merge_r(person_t x,person_t y)
{
	return person_t(1-y.q()/(1-x.q()*y.p()));
};

person_t merge_l(person_t x,person_t y)
{
	return person_t(x.p()/(1-x.q()*y.p()));
};

person_t a[100005],b[100005] ,c[100005];

int main(void)
{
	scanf("%d",&n);
	for(int i=1;i<=n;++i) scanf("%lf",&a[i].p_);
	if(n==3)
	{
	    //something 只看第一个人向左边传还是右边传
	}
	b[2]=a[2];
	for(int i=3;i<=n;++i) b[i]=merge_l(b[i-1],a[i]);
	c[n]=a[n];
	for(int i=n-1;i>=1;--i) c[i]=merge_r(a[i],c[i+1]);
	//ANSWER i=1;
	//ANSWER i=2;
	for(int i=3;i<=n-1;++i)
	{
		if(i+2<=n) p =merge_r(c[i+2],p);
		if(i-2>=2) p =merge_l(p,b[i-2]);
		float ans=0;
		ans+=p.p()*a[i+1].q(),p.q()/(1-p.q()*a[i+1].q());
		ans+=p.q()*a[i-1].p(),p.q()/(1-p.q()*a[i-1].p());
	}
	return 0;
}
\end{minted}
\end{document}
