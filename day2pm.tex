\documentclass{article}
\usepackage{ctex}
\usepackage{amsmath}
\usepackage{amssymb}
\usepackage{minted}
\newcommand{\romannum}[1]{\uppercase\expandafter{\romannumeral#1}}
\begin{document}
\section{Classical Problems}
\subsection{矩阵乘法}
明显地,给矩阵乘法加括号的方案数等于卡特兰数。

另外,我们有以下性质:
\begin{enumerate}
    \item{每一次做矩阵乘法的时候,都是将一个序列内连续区间的矩阵相乘,然后再乘上另外一个区间中的矩阵的乘积,最后再将两个矩阵乘起来。}
    \item{第一个区间中的矩阵乘积的行数等于第二个区间中矩阵乘积的列数。}
\end{enumerate}

定义$dp[i,j]$表示把$[i,j]$这个区间中的矩阵全部乘起来所需要花费的最小代价,则最终答案为$dp[1][n]$,在转移时,枚举一个$k$,表示最后一个乘上去的矩阵是$k$,则我们有以下状态转移方程:
\begin{equation*}
    dp[i,j]=\min\{dp[i,k]+dp[k+1,j]+r[i]\times c[j]\times c[k].\}
\end{equation*}
注意转移时不能按照正常的顺序转移,枚举的应该是区间长度,然后再转移。或者可以使用DFS的方式进行转移。
\begin{minted}{C++}
#include <cstdlib>
#include <cstring>
#include <cstdio>
#include <cmath>
#include <climits>

int n;
int r[1005], c[1005];
int dp_[1005][1005];

void update(int &a, int b) {
    if (a > b) a = b;
}

int dp(int i, int j) {
    if (i == j) {
        return 0;
    }
    if (dp_[i][j] != -1) {
        return dp_[i][j];
    }

    int temp = INT_MAX;
    for (int k = i; k <= j-1; ++k) {
        update(temp, dp(i, k) + dp(k+1, j) + r[i] * c[j] * c[k]);
    }
    return dp_[i][j] = temp;
}

int main(void) {
    memset(dp_, unsigned char(255), sizeof(dp_));
    scanf("%d", &n);
    for (int i = 0; i < n; ++i) {
        scanf("%d%d", r+i, c+i);
    }
    printf("%d\n", dp(1, n));

    return 0;
}
\end{minted}

可以很容易的发现,在递归的顺序或枚举的拓扑序不是十分明显时,使用DFS加上记忆化搜索可以使代码清晰明了。

时间复杂度$O(n^3)$.
\subsection{最优三角剖分(Uva 1626)}

把一个n个顶点的凸多边形剖分成三角形。每一个三角形有唯一的权值函数$w(i,j,k)$。求最优剖分方法,最大化权值和。n不超过100,假设函数$w$定义如下:
\begin{equation*}
    w(i,j,k)=e\cos C+i\sin S.
\end{equation*}
设有两个向量:
\begin{equation*}
    \begin{aligned}
        \alpha&=(x_2-x_1,y_2-y_1)\\
        \beta&=(x_3-x_1,y_3-y_1)\\
        \frac{1}{2}(\alpha\times\beta)&=\frac{1}{2}|(\alpha_1\beta_1-\alpha_2\beta_1)|
    \end{aligned}
\end{equation*}
$\cdots\cdots(其余内容被留作课后作业)$

dp[i,j]表示以i,j两个顶点构成的三角形的最大权值,则有:
\begin{equation*}
    dp[i,j]=
    \begin{cases}
        0&,i=j,\\
        \min\limits_{i\le k\le j}\{t[i,k]+t[k+1,j]+w(v_{i-1}v_kv_j)\}&,i<j.
    \end{cases}
\end{equation*}
\subsection{括号序列}
令$dp[i,j]$表示区间$[i,j)$范围内最少需要补多少个括号。
\begin{equation*}
    dp[i,j]=
    \begin{cases}
        0&,i=j,\\
        1&,i+1=j,\\
        dp[i+1,j-1]&,of\,(s)\,or\,[s],\\
        dp[i,k]+dp[k,j]&,i\le k<j.
    \end{cases}
\end{equation*}
\subsection{区间染色\romannum{2}}
与括号序列有些类似,不妨参照括号序列的状态转移。令dp[i,j]把区间$[i,j)$全部染色所需要的最小代价,则有:
\begin{equation*}
    dp[i,j]=
    \begin{cases}
        0&,i=j,\\
        1&,i+1=j,\\
        dp[i,k]+dp[k,j]&,\textbf{...TODO...}
    \end{cases}
\end{equation*}
\subsection{传球问题\romannum{2}}
假设两个人来回传球,类似等比数列求和公式的处理。
\begin{equation*}
    等比数列求和公式:1+p+p^2+\cdots+p^n=\frac{p^n-1}{p-1},p\neq 1.
\end{equation*}

可以把相互传球的人看成一个整体,因为这些人无论相互传球传多少次,球最终总会被传出这些人的手里。考虑矩阵乘法这道题目,合并$n\times m$的代价等于合并$n\times k$的代价加上$k\times m$的代价。

用类似的方法只考虑两个人,假设右边的人拿到了球向左传球的概率为$p_i$,定义另外一个$q_i=1-p_i$表示向右传球的概率。则球从右边传出去的概率$P=q_j+p_jq_ip_j+p_jq_ip_jq_ip_j+\cdots$,容易看出这是一个等比数列,则从左边传出的概率与从右边传出的概率是一个定值,所以这两个人与一个人是没有区别的。

对于两个点而言,我们已经知道球从左边出去和从右边出去的概率是多少,我们可以合并过来第三个人表示在这两个人右边的人所构成的整体,同样的我们也可以合并过来这两个人左边的人,问题就转换成四个人的问题。上述合并过程可以使用递归来实现。

现在我们需要设法求解一下四个人的问题:

(弃疗……)
\end{document}
