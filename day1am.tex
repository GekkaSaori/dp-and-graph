\documentclass{article}
\usepackage{ctex}
\usepackage{amsmath}
\usepackage{minted}
\begin{document}
\title{Note on Dynamic Programming}
\section{Introduction}
\subsection{数字三角形}
“拉”法:$dp[i,j]=\max\{dp[i-1,j-1],dp[i-1,j]\}+value[i,j]$

“推”法:
\begin{equation*}
\begin{aligned}
&dp[i,j]+value[i+1,j]&\to&dp[i+1,j]\\
&dp[i,j]+value[i+1,j+1]&\to&dp[i+1,j+1]
\end{aligned}
\end{equation*}
\subsubsection{DP与图论的关系}
迷宫问题,求解S-T的最短路。

考虑搜索的方案。
\begin{minted}{python}
    global ans = infinity
    procedure Search[pos, temp]:
        if pos == T:
            ans = max[ans, temp]
        else:
            for (pos, next_pos, cost) in graph:
                Search[next_pos, temp + cost]
\end{minted}

搜索算法有什么特点?第一次到达某个点,就是最优的解。

尝试改进搜索的算法。

\textbf{拓扑排序}。以图论的思想考虑这个问题,会发现DP方程与图上最短路问题有些类似。每一个点看做具有一个点权,最大化得到的点权值(以状态转移过程方向连边)。与拓扑排序相近。

如果这张图不能被拓扑排序,则说明状态设计错了(如果不能理解,参见CLRS上动态规划一章状态转移图是拓扑图的解释)。

单源最短路问题(伪代码采用py的语法高亮):
\begin{minted}{python}
global ans = infinity
procedure Search(pos, temp):
    if pos == T:
        ans = max(ans, temp)
    else:
        for(pos, next_pos, cost) in graph:
            Search(next_pos, temp + cost)
\end{minted}
事实上就是把所有的状态画到一张纸上,然后按照状态转移的方向连成一张图,发现是一个拓扑图。然后需要按照拓扑图上的方向进行转移,即是一个动态规划的思路。

拓扑排序:所有在到达某个点前到达的点被排在这个点的前面。充要条件:图是一个有向无环图(DAG---Directed Acyclic Graph)。树上DFS时产生的DFS序就是一个拓扑序。时间复杂度$O(V+E)$.

迷宫问题显然不能拓扑排序。但是仍然可以使用BFS等方法完成。能否使用类似的方法优化这个问题?

就是所有的决策都是由内圈扩展到外圈的过程,就可以使用DP。当边权都为正的时候,就能由内圈扩展到外圈(明显地,边权不可能变小,即不可能出现经过某些额外的节点可以使得边权更小),按照到S的距离使用堆排序,Dijkstra算法(就是一个贪心的思想,如果不能理解,参见CLRS图算法最短路径部分)。

以上的问题统一称为线性规划问题(包括最短路问题),请参见CLRS,我不会。

遇到负边权怎么办?Dijkstra算法不能使用了。可能需要重新设计一下状态:
\begin{enumerate}
\item{Bellman-Ford算法}dp[i,x]表示最多经过x个点,到达点i所需要的最短路径长度。则有以下状态转移方程:
\begin{equation*}
dp[i,x]=\min\{dp[i-1],x,min_p\{dp[i-1,p]+cost(p,x)\}\}
\end{equation*}
\item{Floyd算法}
\end{enumerate}
\subsection{状态设计}
核心是最优子结构,或者可以称为“封闭子结构”,到达一个状态的方法,和之后的决策是已经无关的。

距离,TSP问题(旅行商问题,可以参考CLRS)。
\subsubsection{矩形嵌套问题}
设计状态发现,只需要考虑最后一个矩形是i时,最多已经排了多少个矩形:
\begin{minted}{python}
function Search(i):
    ans = 1
    for j = 1 to n:
        if CanInlay(j,i):
            ans = max(ans, Search(j) + 1)
            Dp[i] = ans
            return ans
\end{minted}
\subsubsection{计数问题}
骨牌覆盖问题:有一个2行n列的长方形网格,要求用n个1*2的骨牌铺满,求覆盖方案。

是否存在一种交错的方案?可以证明不可能出现交错的情况。

考虑以下做法:

假设第i个位置是纵向骨牌,那么有以下递推式:
$f(n)=f(0)f(n-1)+f(1)f(n-2)+\cdots+f(n-1)f(0)$

这个做法是错误的。问题如下:
\begin{enumerate}
  \item{没有纵向的骨牌}
  \item{枚举纵向骨牌出现在何处,有些方案(出现两个纵向骨牌)会被计算两次}
  \begin{itemize}
    \item{考虑最早枚举的纵向骨牌出现的位置}
    \item{假设第i个位置是第一个纵向骨牌,他的左边全部都是横向骨牌}
    \item{分两种情况讨论:}
    \begin{itemize}
      \item{偶数:$f(n)=f(n-1)+f(n-3)+f(n-5)+\cdots+f(1)$}
      \item{奇数:$f(n)=f(n-1)+f(n-3)+f(n-5)+\cdots+f(0)$}
    \end{itemize}
  \end{itemize}
\end{enumerate}
\subsubsection{凸多边形划分问题}
\begin{itemize}
    \item{解法1:考虑$v_1v_n$这条边所在的三角形的位置,则有:
    \begin{equation*}
      f(n)=f(2)f(n-1)+f(3)f(n-2)+\cdots+f(n-1)f(2)
    \end{equation*}
    每次把多边形划分成两个凸多边形。}
    \item{解法2:考虑$v_1v_k$这条对角线,假设连了$v_1v_k$这条对角线,则会发现一边是$n-k$的多边形,另一侧是一个$n-k+2$的多边形。
    \begin{equation*}
      f(3)f(n-1)+f(4)+f(n-2)+\cdots+f(n-1)f(3)
    \end{equation*}
    每一个点出发的对角线都有上述的效果,所以枚举每一个点出发的每一条对角线,共计有n个点:
    \begin{equation*}
      n[f(3)f(n-1)+f(4)+f(n-2)+\cdots+f(n-1)f(3)]
    \end{equation*}
    每一条对角线的效果会被计算两次(明显地,枚举每一个点,$v_1v_k$这条对角线会在$v_1$这个点被枚举一次,在$v_k$这个点又被枚举一次):
    \begin{equation*}
      \begin{aligned}
        f(n)&=n[f(3)f(n-1)+f(4)+f(n-2)+\cdots+f(n-1)f(3)]/[2(n-3)]\\
        &=>f(n+1)=f(n)*(4n-6)/n
      \end{aligned}
    \end{equation*}
    }
\end{itemize}
其中解法2是一种计数问题中非常常见的处理方法,我们发现每种情况被算了多次,那么就除以这种情况被算的次数即可(类似排列组合的公式的推导)。

以上技巧称为``Double-counting'',以上递推公式就是Catelan数的定义。

常见应用还有括号序列,给定一个长度,问在这个长度下有多少个合法的括号序列。

出栈顺序、乘法方案数等等。

\end{document}
