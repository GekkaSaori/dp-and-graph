\subsection{最长公共子序列}
令dp[i][j]表示以第一个串的第i位与第二个串的第j位结尾的最长公共子序列长度。

状态的转移有三种:
\begin{equation*}
    dp[i][j]=\max
    \begin{cases}
        dp[i-1,j]\\
        dp[i,j-1]\\
        dp[i-1,j-1]+1,a[i]=b[j].
    \end{cases}
\end{equation*}
\subsection{最长公共上升子序列}
\subsubsection{三维DP}
令dp[i,j,k]表示以第一个串的第i位与第二个串的第j位结尾,最后一位是k的最长公共上升子序列的长度,则有状态转移方程:
\begin{equation*}
    dp[i,j,k]=\max
    \begin{cases}
        dp[i-1,j,k]\\
        dp[i,j-1,k]\\
        dp[i-1,j-1,k'], k<k'.
    \end{cases}
\end{equation*}

\subsubsection{降维优化}
dp[i,j]表示以a串的第i位结尾,b串的第i-j位结尾的LCS,且最后元素为b[j],则有:
\begin{equation*}
    dp[i,j]=\max
    \begin{cases}
        dp[i-1,j]\\
        dp[i-1,j'],&j'<j,\\
        &b[j']<b[j],\\
        &a[i]=b[j].
    \end{cases}
\end{equation*}

\section{区间模型}
\subsection{矩阵乘法}
矩阵:一个$N\times M$的矩阵被定义为一个N行M列的数组,数组中的每个元素都是一个实数。一个向量可以表示为一个$M\times 1$的矩阵。

应用:

矩阵乘向量:
\begin{equation*}
    \begin{bmatrix}
        a_{11}&a_{12}\\
        a_{21}&a_{22}
    \end{bmatrix}
    \times
    \begin{bmatrix}
        x_{11}\\
        x_{21}
    \end{bmatrix}
    =
    \begin{bmatrix}
        y_1\\
        y_2
    \end{bmatrix}
\end{equation*}
有一个任意的向量$x(x_0,y_0)$旋转\theta\textdegree,则有:
\begin{equation*}
    \begin{aligned}
        x_1=x_0\cdot\cos\theta-y_0\cdot\sin\theta\\
        y_1=x_0\cdot\sin\theta+y_0\cdot\cos\theta
    \end{aligned}
\end{equation*}

矩阵乘法:
定义矩阵乘法运算规则如下:
\begin{equation*}
    \begin{aligned}
        C[i,j]&=\sum_{k=1}^m A_{ik}B_{kj}\\
        &=A_{i1}B_{1j}+A_{i2}B_{2j}+\cdots +A_{im}B_{mj}
    \end{aligned}
\end{equation*}
时间复杂度为$O(n\times m\times k)$.

矩阵乘法与DP的关系:矩阵乘法优化递推转移。

例题:斐波那契数列升级版

题解:矩阵快速幂优化递推转移。

(抱歉,Atom崩了,推导过程没写,具体过程可参照题解)
