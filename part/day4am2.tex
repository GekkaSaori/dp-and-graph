\documentclass{article}
\usepackage{ctex}
\usepackage{amsmath}
\usepackage{amssymb}
\usepackage{minted}
\usepackage[colorlinks, linkcolor=blue]{hyperref}
\usepackage{graphicx}
\newcommand{\romannum}[1]{\uppercase\expandafter{\romannumeral#1}}
\setminted[C++]{autogobble,breaklines,fontsize=\footnotesize,mathescape}
\begin{document}
\title{Note on Dynamic Programming (\romannum{4})\\\large{Day 4 AM---Optimization and Others}}\date{}\author{Hatsune Miku}
\maketitle
\section{Optimization and Others}
\subsection{矩阵优化}
\subsubsection{骨牌覆盖}
有$4\times N$的棋盘,用$1\times 2$的骨牌覆盖,求覆盖的方案数$\bmod 10^9+7$的值,$N\le 10^9$.

用$f[i,s]$表示前$i-1$列已经被完全覆盖,且第$i$列的覆盖情况是$s$,如s=\verb+1001+表示第一格和第四格已被覆盖,中间两格未被覆盖。

枚举在s到s'的转移(使用dfs),我们可以得到以下递推式:
\begin{equation*}
    \begin{aligned}
        f[i+1,s']=\sum_{j=1}^nf[i,j]\\
        f[i+1,0]=\sum_{j=1}^nf[i,j]\\
    \end{aligned}
\end{equation*}
由$f[i+1]=Af[i]$我们有:
\begin{equation*}
    f[n+1]=A^n-f[i]
\end{equation*}
使用矩阵快速幂优化转移。

矩阵优化本质上是为了解决$f[i,j]$转移到$f[i+1,j']$中,如果对于j这一维可以看成是上一行的若干个点求和转移到下一个点时,可以使用矩阵优化。
\subsection{单调队列}
\subsubsection{篱笆}
篱笆由N个连续的木板构成,有K个工人染色,每块木板只能被染一次,第$i$个工人染一个木板赚$cost[i]$,要么不染色,要么染一个连续的区间,区间必须包含$pos[i]$,区间的最大长度为$len[i]$,求总共最多赚多少钱。$N\le 20000, K\le 100.$

令$f[j]$表示前j块已经染完的最优方案,我们有:
\begin{equation*}
    f[i,j]\leftarrow f[i,k]+cost(i).
\end{equation*}
扩充一下状态$f[i,j]$表示前i个工人已经决定了,前j个木板也决定了。在这种情况下最优方案是什么。答案是$f[k,n]$,每次枚举最后一个人怎么染。
\begin{equation*}
    \begin{aligned}
        f[i,j]=\max
        \begin{cases}
            f[i,j-1],\\
            f[i-1,j],\\
            f[i-1,k]+(j-k)\times cost(i)
        \end{cases}
        ,j-len(i)\le k且k<pos(i).
    \end{aligned}
\end{equation*}
考虑排序顺序的问题。这个题是可以按pos(i)进行排序的,因为我们可以发现一个区间的左端点在左边,右端点在右边,在此情况下按右端点排序是错误的。应该按照线段的中心点进行排序,时间复杂度为$O(n^2k)$,是不能通过这道题目的。

考虑化简一下递推式:
\begin{equation*}
    f[i,j]=\min\limits_k\{f[i-1,k]-k\times cost[i]\}+j\times cost[i].
\end{equation*}
我们可以看到$k\times cost[i]$考虑维护一个数列$A_k=f[i-1,k]-k\times cost[i]$,对它求最值,明显地数列$A_k$与$j$无关。这个数列的左端点是固定的,右端点是单调递增的,且数列满足性质$j-cost[i]\le k\le pos(i)$,很明显这是一个单调队列(明显的,$k$满足性质$L[i]\le k\le R[j]$,其中$L[i]$与$R[j]$这个区间范围是单调不下降的,满足单调队列的性质。参考“滑动窗口”这道题目,维护一个队列,如果满足$i<j$且$A_i<A_j$,那么$A_i$就再也不会出现了,可以从队列中删除,就得到了一个单调队列)。

代码实现:
\begin{minted}{C++}
int queue[10005], head=0, tail=0;
void push(int x) {
    while (head < tail) {
        if (a[queue[tail - 1]] < a[x]) {
            --tail;
        }
    }
    queue[tail++] = x;
}

void pop(int x) {
    while (head < tail) {
        if (q[head] < x) {
            ++head;
        }
    }
}

int last = 0;
for (int j = 0; j < m; ++j) {
    for (int k = last; k <= R[j]; ++k) push(k);
    pop(L[j]);

    printf("%d\n", queue[head]);

    last = R[j];
}
\end{minted}

可以解决给定一个数组,有若干个询问,每次询问区间内的最值是什么,区间单调向右的问题。

回来看这道题,只需要用单调队列维护$f[i-1,k]-k\times cost[i]$,就能够在$O(n)$的时间内解决查询k大的问题,最终总的时间复杂度为$O(nk)$.
\end{document}
