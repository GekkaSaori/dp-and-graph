\documentclass{article}
\usepackage{ctex}
\usepackage{amsmath}
\usepackage{minted}
\begin{document}
\section{讲题大会}
\subsection{T1 环}
这个题目有两种做法,快速幂、倍增。

快速幂,二进制优化。首先由于乘法具有结合律,所以我们有:
\begin{equation*}
    a^{13}=(\underbrace{a\times a\times\cdots\times a}_{8个a})(\underbrace{a\times a\times\cdots\times a}_{4个a})(a)
\end{equation*}
所以我们可以对13这个数进行二进制分解,分解成$(1101)_B$的形式。最高位的1看做$a^3$,第二位的1看做$a^2$,最后一位的1看做$a^0$,然后把它们一一乘起来。

然后可以以$O(\log r)$的复杂度计算$n^r$.

快速幂取模,因为乘法和取模可以分配,所以放到快速幂运算过程中取模。特别要注意精度问题,如果是对$10^9+7$取模,需要先转换成\verb+long long+的形式然后再做。

本题本身是个类似函数的复合的题目,可以使用类似快速幂的思想去计算。函数的计算必须从内向外做,所以要注意计算的顺序的问题。

对m进行二进制分解,假设需要跳13层,利用快速幂,从内层向外层跳1层+4层+8层。

注意与快速幂有一点不同,需要使用倍增算法进行计算。例如$f^2(x)=f(f(x))$,$f^4(x)=f(f^2(x))$,$f^8(x)=f(f^4(x))$,利用这个递推性质,我们就可以利用倍增算法计算任意函数值了。

类似的有关做法:考虑有一个排列$(1,3,2,5,4)$,有$f(1)=1,f(2)=3,f(3)=2,f(4)=5,f(5)=4$,把它们看做一个映射,用一个有向的箭头把$x$与函数值连接起来,可以发现,用这种方式表示一个排列,一定构成了若干个环,证明思路是双射+不重复的环,所以可以运用一个类似DFS的操作预处理出排列中所有的环。现在需要计算$f^m(x)$,假设这个环的长度为$k$,所以我们只需要等价的移动$m\bmod k$步,再利用数组下标映射一下。例如需要计算第i个元素的位置,那么可以计算$(i+m\bmod k)\bmod k$。

读入的方法:设读入的行数为m,已知$m=n+2q$,若$m$为奇数,则第一行一定是排列中的一个数。剩下的就是偶数的情况,两个两个的检查第一个数是不是在之前出现过,如果出现过就说明它一定是开始询问的部分,因为数据保证一定合法。
\subsection{T2 礼}
思路基本上就是一个折半搜索。我们有一个集合$a$,要从中选择一个子集$b$,使得$b\subseteq a$。显然我们有一个暴力的想法,直接枚举$a$的每一个子集,可以通过50\%的数据。

考虑全部的数据,能否有一个$O(n^4)$的算法,不太可能。那么有没有一个$O(2^{\sqrt n})$的算法?答案就是这样。

利用meet-in-the-middle的算法,我们枚举集合$s_1$,使得$s_1\subseteq a的左半部分$,类似地枚举$s_2$,最终的答案是$sum(s_1+s_2)\bmod m$。我们可以分两部分讨论,若$s_1+s_2<m$则只需要找一个尽可能大的值;考虑当$s_1+s_2\le m$的情况,如何才能使得$s_1+s_2\bmod m$最大?

预处理出$s_2$的所有取值,一共有$2^{17}$个数,使得$s_2<m-s_1$并且$s_2$尽可能大,所以可以对$s_2$进行排序,然后二分查找一下最大值。

时间复杂度$O(2^{17}\log 2^{17})$.

骗分的手段:背包。由于a[i]不是特别的大,所以可能能骗到很多分数。假设随机生成一组a[i],则最大值一般就是$M-1$,所以在数据小的情况下,总是要取几个数。
\subsection{T3 变}
考虑把整个数轴以$a_x$为单位长度划分,则每次跳跃都会跳到距离$a_x$最近的点上。

有几点观察:
\begin{enumerate}
    \item{贪心的思想,每次操作,都需要让A减少值尽量大。

    如果某一步有两种走法,那么令A减少值比较大的数的方案在下一步无论进行什么操作,可以到达的位置一定不会比另外一种方案大,即这种方法可以更快地到达B。}
    \item{如果$A-A\bmod a_x$,则说明这个$a_x$再也不会被用到了,因为对于这个$a_x$,跳$a_x$步所到达的点一定小于B,这样就永远也无法到达B这个点了。}
\end{enumerate}
根据上述两点观察我们即可得出std的代码了。\verb+std::set+支持上述两种操作。

时间复杂度的证明:

考虑在连续若干次迭代的过程中,假设集合$a$的大小不变且为$n$,则在连续三次枚举后,A的值至少会减少$n$.

$a$这个集合内没有重复元素,且$a$中存在一个$a_{\max}$,则$a_{\max}\ge n$.

如果利用这个$a_{\max}$三次,则至少可以让A减少$n$,因为根据上述的贪心,连续三次操作后,至少跳跃了一个$a_{\max}$的值。

所以进行$O(n)$的枚举,可以保证A的值一定可以减小$n$。因为A最多可以减少A-B次,所以可以保证时间复杂度为$O(A-B)$这个量级。

考虑$a$这个集合发生变化的情况,由于满足每个元素仅在集合中出现一次,所以每个元素最多可以被踢掉一次,每次踢元素重新枚举带来的额外的时间消耗为$O(n)$,所以总的时间复杂度为$O(A-B+n)$。

如果使用\verb+std::set+,时间复杂度加一个$n\log n$,是$O(A-B+n\log n)$,也是可以通过所有的数据的。

注意一定需要去重,否则贪心的性质不成立,因为可能有多个$a_{\max}$,并且不去重时间复杂度也无法保证。
\end{document}
