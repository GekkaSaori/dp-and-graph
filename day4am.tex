\documentclass{article}
\usepackage{ctex}
\usepackage{amsmath}
\usepackage{amssymb}
\usepackage{minted}
\usepackage[colorlinks, linkcolor=blue]{hyperref}
\usepackage{graphicx}
\newcommand{\romannum}[1]{\uppercase\expandafter{\romannumeral#1}}
\DeclareMathOperator{\bor}{or}
\setminted[C++]{autogobble,breaklines,fontsize=\footnotesize,mathescape}
\begin{document}
\title{Note on Dynamic Programming (\romannum{3})\\\large{Day 3 AM to Day 3 NIGHT---Advanced Models}}\date{}\author{Hatsune Miku}
\maketitle
\section{Advanced Models}
\subsection{数位模型}
\subsubsection{数位翻转}
有给定$L\le R$,求区间$[L,R]$中,有多少个数翻转后仍然在$[L,R]$中。$L, R\le 2^{64}$。

考虑到我们写数字的一般形式,是一位一位按照数位书写的。现在我们需要可以考虑按照数位进行DP。

令$f[i]$表示已经考虑了这个数字的前i位,有多少种方案数。先不用管倒过来还在这个区间里这个限制条件,考虑如何转移。假设原来的数字是$l_1l_2l_3l_4$和$r_1r_2r_3r_4$,合并后需要满足前缀大于$l$,小于$r$.

可以使用“推”的方法思考这道题目。因为只使用$f[i]$这一维无法记录前面的位数,如果存在一个前面的位数,使得当前这个数小于$r$,那么后面的数字怎么填就无关紧要了(显然)。所以除了$f[i]$这一个数组外,我们还需要知道这个数之前填的是什么。

用$c_1$表示当前这一位与l的关系。同理用$c_2$表示当前这一位与$r$的关系。$c_1$与$c_2$有两种状态:
\begin{equation*}
    \begin{aligned}
        c_1=
        \begin{cases}
            \verb+'='+,当前这一位等于l,\\
            \verb+'>'+,当前这一位大于l.
        \end{cases}
        &
        c_2=
        \begin{cases}
            \verb+'='+,当前这一位等于r,\\
            \verb+'<'+,当前这一位小于r.
        \end{cases}
    \end{aligned}
\end{equation*}

接下来考虑如何使状态的转移符合第二个条件。考虑把一个数翻过来以后与l的最后若干位和r的最后若干位关系。再定义一个$c_3$和$c_4$表示。需要考虑三种情况,小于、等于和大于。

以上代码默认了两个条件,两个数字位数相同,如果需要处理需要翻转的情况,可以在外面直接枚举答案的位数$n$,还有如何统计答案的问题,假设l的长度为$n_l$位,$r$的长度为$n_r$位,则可以直接暴力枚举$c_1$, $c_2$, $c_3$, $c_4$的符号即可。

一般套路是令$f[i]$表示已经考虑了这个数字的前i位,有多少种方案数。然后考虑在L和R区间内数字的大小关系即可。

另一个问题,求区间$[L,R]$内有多少个数不存在连续的两个相同的数码。
\begin{minted}{C++}
f(i, c1, c2, c3, c4)
->f(i+1, d1, d2, d3, d4)

int get_next_f(int c1, int k, int r) {
    int d1;
    if (c1 == '=') {
        if (k == r) d1 = '=';
        else if (k < r) d1 = '<';
        else d1 = '>';
    } else {
        d1 = c1;
    }
    return d1;
}

int get_next_b(int c3, int k, int r) {
    int d3;
    if (c3 == '=') {
        if (k == r) d3 = '=';
        else if (k < r) d3 = '<';
        else d3 = '>';
    } else if (c3 == '>') {
        if (k == r) d3 = '>';
        else if (k < r) d3 = '<';
        else d3 = '>';
    } else {
        if (k == r) d3 = '<';
        else if (k < r) d3 = '<';
        else d3 = '>';
    }
    return d3;
}

// for i, c1, c2, c3, c4
{
    for (int k = 0; k < 10; ++k) {
        int d1 = get_next_f(c1, k, L[i+1]);
        int d2 = get_next_f(c2, k, R[i+1]);
        int d3 = get_next_b(c3, k, L[n-i]);
        int d4 = get_next_b(c4, k, R[n-i]);
        dp[i+1][d1][d2][d3][d4] += dp[i][c1][c2][c3][c4];
    }
}

// for n
{
    // dp()
    // for c1 c2 c3 c4 {
    ans += dp[n][c1][c2][c3][c4];
}
}

// T2

// for i, j, c1, c2
{
    for (int k = 0; k < 10; ++k) {
        if (j != k) {
            int d1 = get_next_f(c1, k, L[i+1]);
            int d2 = get_next_b(c2, k, R[i+1]);
            dp[i+1][k][d1][d2] += dp[i][j][c1][c2];
        }
    }
}

// for n
{
    dp()
    // for j, c1, c2 {
    ans += dp[i][j][c1][c2];
}
}
\end{minted}
\subsubsection{翻转求和}
R(x)为x数码倒过来得到的数,问有多少个x满足:$x+R(x)=N$,$N\le 10^{10000}.$

跟上道差不多,不用枚举大于号小和等于号。
\subsubsection{组合数计数(BZOJ4737)}
Lucas定理。
\end{document}
